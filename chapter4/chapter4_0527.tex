\documentclass[a4paper]{ltjsarticle}

\usepackage[deluxe,noto-otf,no-math]{luatexja-preset}
\usepackage{luatexja-otf}
\usepackage{graphicx}
\usepackage{listings}
\usepackage{amssymb,amsmath}
\usepackage[margin=20truemm]{geometry}
\usepackage{siunitx}
\usepackage{here}
\setmainfont[BoldFont=Noto Serif CJK JP Bold]{Noto Serif CJK JP Regular}
\setsansfont[BoldFont=Noto Sans CJK JP Bold]{Noto Sans CJK JP Regular}


\title{第15回輪講資料\\
「コンピュータ・ネットワーク」\\
第4.3節 イーサネット(pp.272-282)}
\author{岡崎 雅大 <Okazaki Masahiro>\\
okazaki@nile.cse.kyutech.ac.jp}
\date{2019年5月27日(月)}

\begin{document}
\maketitle
\tableofcontents
\section{はじめに}\label{ux306fux3058ux3081ux306b}

本稿では,教科書「コンピュータネットワーク」より,第4章4.3節4.3.1項から4.3.4項までの内容をまとめたものである.
これまでに述べたチャネル割当プロトコルが,実際のシステムへどのように適用されているかについて述べていく.
とくに本稿では世界でもっとも広く普及しているコンピュータネットワークであろう,イーサネットについて取り扱う.

\section{イーサネット}\label{ux30a4ux30fcux30b5ux30cdux30c3ux30c8}

イーサネットには \textgt{\textbf{クラシック・イーサネット(classic Ethernet)}}
と \textgt{\textbf{スイッチ式イーサネット(switched Ethernet)}} の2種類がある.
クラシック・イーサネットはこの章で触れる技術を用いて,マルチ・アクセスにより生じる問題を解決している.
スイッチ式イーサネットはコンピューター間をつなぐために「スイッチ」と呼ばれるデバイスを用いる.
これらの両方が「イーサネット」と呼ばれるが,両者は大きく異なっている.
クラシック・イーサネットはイーサネットのもともとの形態である.
スイッチ式イーサネットはイーサネットの現在の形である.
ここからは,イーサネットの歴史的な変遷をそれらがどのように開発されたと共に時系列に沿って論じていく.

\subsection{クラシック・イーサネットの物理層}\label{ux30afux30e9ux30b7ux30c3ux30afux30a4ux30fcux30b5ux30cdux30c3ux30c8ux306eux7269ux7406ux5c64}

イーサネットの歴史はALOHAの歴史と同時に始まった. Bob
Metcalfという学生がM.I.Tで学士号を取得し,博士号のコースをハーバード大学で歩み始めたときである.
研究を進める中で,彼はAbramsonの研究に出会った.
その後,彼はハーバードを卒業しXerox PARC(Palo
Alto研究所)で働き始めた.
そこで彼が目にしたものは,後に「パーソナル・コンピューター」と呼ばれるもの,Xerox
Altoであった. ただし,個々の機械は互いに独立していた.
彼は,Abramsonの研究に関する知識を用いて同僚のDavit
Boggsとともに,はじめてのLANを設計・実装した.
伝送媒体として同軸ケーブルが用いられ,速度は3Mbpsであった.
彼らはかつて電磁放射を伝えると考えられていた発光性エーテルにちなんで,このシステムを
\textgt{\textbf{イーサネット(Ethernet)}} と呼んだ.

Xeroxのイーサネットが大成功したので,DECとIntel,そしてXeroxは
\textgt{\textbf{DIX規格(DIX Standard)}} と呼ばれる10Mbpsの規格を制定した.
このDIX規格に小さな変更を加えることで,1983年にIEEE802.3という規格が作られた.
Xeroxは,すでにパーソナル・コンピューターのような独創的発明をしていたにもかかわらず,この規格の商業化に失敗した.
Xeroxが規格の標準化以外にほとんど興味を示さなかったため,Metcalfeはイーサネット・アダプターを販売する会社,3Comを設立し,アダプターを何百万個も販売した.

クラシック・イーサネットでは,一本の長いケーブルが建物中を這い回り,すべてのコンピューターが接続された.
このアーキテクチャを図4-13(p274)に示す.

\begin{itemize}
\item
  \textgt{\textbf{シック・イーサネット(thick Ethernet)}}

  \begin{itemize}
  \item
    ケーブルは黄色い庭のホースに似ている(規格として色が指定されているわけではないが,黄色が推奨されている)
  \item
    2.5mごとにコンピューターを接続する場所を示す印が示されている
  \item
    1つのセグメントの長さは\SI{500}{m}以内
  \item
    接続台数は100台以内
  \end{itemize}
\item
  \textgt{\textbf{シン・イーサネット(thin Ethernet)}}

  \begin{itemize}
  \item
    より曲げやすく,工業規格であるBNCコネクタを用いる
  \item
    安価かつ導入が安易である
  \item
    1つのセグメントの長さは\SI{185}{m}以内
  \item
    接続台数は30台以内
  \end{itemize}
\end{itemize}

どちらのイーサネットを用いても信号伝搬距離からセグメントあたりの長さの上限が決まっているため,より大規模なネットワークを構築するためには,複数のケーブルを
\textgt{\textbf{リピーター(repeater)}} と呼ばれる装置で接続する必要がある.
リピーターは,信号を受信・増幅し,その信号を機器の両方向に返す物理層のデバイスである.
ソフトウェアから見ると,リピーターで接続した複数のケーブルは一本のケーブルと同じように見える.

これらのケーブルにより,情報はマンチェスタ符号化を用いて送信される(マンチェスタ符号化については教科書2.5節を参照).
イーサネットは複数のケーブルで構成することができるが,どのトランシーバー間も\SI{2.5}{km}以上離すことはできないほか,2つのトランシーバー間に4つ以上のリピータを挟むこともできない.
これらの制限はMACプロトコルを正常に動作させるためのものである.

\subsection{クラシック・イーサネットのMAC副層プロトコル}\label{ux30afux30e9ux30b7ux30c3ux30afux30a4ux30fcux30b5ux30cdux30c3ux30c8ux306emacux526fux5c64ux30d7ux30edux30c8ux30b3ux30eb}

送信フレームのフォーマットを図4-14(p274)に示す.

\begin{itemize}
\item
  \textgt{プリアンブル(8バイト)}\\
  1〜7バイトはいずれも\texttt{10101010}というビットパターンを持つ.
  最終バイトだけは最後の2ビットが\texttt{11}となる.
  この最終バイトはIEEE802.3でフレーム開始区切り(Start of Frame
  delimiter)と呼ばれている.
  このパターンを符号化すると長さ\SI{6.4}{\micro s}の\SI{10}{MHz}の矩形波となり,これによって送受信者間のクロックを同期させる.また,最後の2ビットによって残りのフレームを送信することを受信者に伝える.
\item
  \textgt{あて先アドレス,送信元アドレス(6バイト)}\\
  あて先アドレスで最初に送られるビットは,通常のアドレスの場合0,グループ・アドレスの場合1である.
  グループ・アドレスにより,1つの場所から送られたものを複数の端末で受信できるようになる.
  フレームがグループ・アドレス宛てに送られると,グループ内のすべての局がそのフレームを受信する.

  \newpage
  \begin{itemize}
  \item
    マルチキャスト(multicasting)\\
    グループ・アドレスに送信することをマルチキャストという.
    ブロードキャストに比べて選択的であるが,各局とグループの対応の管理が必要となる.
  \item
    ブロードキャスト(bloadcasting)\\
    特別なアドレスとして,あて先アドレスの全ビットがすべて1のものはブロードキャスト用として予約されている.
    このフレームは,ネットワーク上のすべての局に送信される.
    局同士を区別する必要がないため,グループ分けの管理も不要である.
  \end{itemize}

  局が持つ送信元アドレスは,IEEEによってどの局間でも重複が起こらないように割り振られており,全世界で一意である.
  アドレスの最初の3バイトは \textgt{\textbf{OUI(Organizationally Unique Identifier)}} に用いられている.
  この値はIEEEにより割り振られ,メーカーを示す.
  メーカーは残りの3バイトを自ら割り振り,NICに登録する.
\item
  \textgt{タイプ,長さ(2バイト)}\\
  フレームがイーサネットかIEEE802.3であるかに依存して内容が異なる.

  \begin{itemize}

  \item
    イーサネット : タイプ・フィールド\\
    受信者にこのフレームをどのプロセスに渡すべきかを明示する.
    しかし,これではデータの中身を見なければフレームの長さがわからないため,階層化の決まりに反してしまう.
  \item
    IEEE802.3 : 長さフィールド\\
    フレーム長を保持している.
    受信者にとってフレームをどう処理すべきか判断する方法は,
    \textgt{\textbf{LLC(Logical Link Control : 論理リンク制御)}}
    \footnote{MAC副層の違いを吸収し,上位層にサービスを提供する層}
    プロトコル用の別のヘッダーをデータ内に追加することで提供される.
  \end{itemize}

  IEEE802.3規格が制定されたとき,すでに多くのDIX規格に基づいた製品が登場していたため,フォーマットの更新はほとんど進まなかった.
  結局IEEEは双方のフィールドを規格として正しいとし,フィールドの値が0x600(1536)以下の場合は長さフィールド,0x600より大きい場合はタイプフィールドとして解釈することとした.
\item
  \textgt{データ(0-1500バイト),パッド(0-46バイト)}\\
  データは1500バイト以下である.
  この制限は,イーサネット標準が制定されたとき,トランシーバーがフレームを格納するのに十分なRAMを必要とし,かつ,当時RAMは高価であったことに基づいてある程度恣意的に選ばれたものである.
  最大フレーム長に加え,最小フレーム長も存在する.

  \begin{itemize}
  \item
    最小フレーム長

    \begin{itemize}
    \item
      衝突を検知したとき,送信中のフレームの送信は打ち切られるため,バラバラとなったビットとフレームのかけらがいくつもケーブル上に存在してしまう.
      もしフレーム長が極端に短いフレームが送信されると,これらと見分けがつきにくい.
    \item
      送信した最初のビットがケーブルの終端へ到着する前にフレームの送信が終了してしまうと,送信側が衝突を検知する前に送信が正常に終了したと誤認する恐れがある.
      そのため,衝突を検知したときにフレームを送信しておくためには,送信時間を終端までの遅延時間の2倍以上とする必要がある.
    \item 
      上記を踏まえ,規格として最小フレーム長は64ビットと定められている.
    \end{itemize}
  \item
    パッド\\
    フレームが終端に到着する遅延時間を\(\tau\)とする.
    上記より,すべてのフレームは送信に\(2\tau\)以上の時間をかけなければならない.
    \SI{2.5}{km}の最長距離と4つのリピータを持つ約10Mビット/秒のLANでは,リピータの遅延時間を含めラウンド・トリップ時間は最悪の場合で約\SI{50}{\micro s}と算出される.
    1ビットの転送に\SI{100}{\nano s}かかるため,最小のフレーム長は500ビットとなる.
    いくらかの余裕を付加して,512ビット(64バイト)に丸められる.
    データフィールドまでのフィールドの長さを合計すると,22バイトであるため,差し引き46バイト必要である.
    よって,データフィールドが46バイト未満であるとき,不足分をパッドフィールドで埋め合わせる.
  \end{itemize}
\item
  \textgt{チェックサム(4バイト)}\\
  フレームのビットが正しく受信されたか判断するための誤り検出コードとして用いられている.
  誤りが検出された場合,そのフレームを破棄する.
  32ビットCRCであり,生成多項式を用いて定義されている(詳細は3.2節を参照).
\end{itemize}

\subsubsection*{べき乗バックオフを用いたCSMA/CD}\label{ux3079ux304dux4e57ux30d0ux30c3ux30afux30aaux30d5ux3092ux7528ux3044ux305fcsmacd}
\addcontentsline{toc}{subsubsection}{べき乗バックオフを用いたCSMA/CD}

クラシック・イーサネットは1-永続CSMAアルゴリズムを用いている.
送信すべきフレームを持っている局が媒体を監視し,媒体が使用されなくなるとすぐにフレームを送信する.
局は送信時にチャネル上の衝突を監視する.
衝突が発生した場合,局は短いジャム信号を発して送信を中止し,ランダムな時間間隔のあとに再送信を行う.
衝突が発生したとき,ランダムな時間間隔をどのように決定するかを考える.

\begin{itemize}
\item
  環境\\
  図4-5(p262)のモデルを用いる.
  衝突が発生すると,時間は離散的なスロットに分割される.
  スロットの長さは,イーサネットの最長往復伝搬時間(\(2\tau\))である.
  よって,スロット時間は512ビット時間すなわち\SI{51.2}{\micro s}である.
\item
  流れ

  \begin{enumerate}
  \item
    衝突が発生すると,局はフレームの送信を中止する.
  \item
    それぞれの局は0または1スロット時間の中から待ち時間をランダムに選択し,待機したあと再送信する.
  \item
    双方が同じ待ち時間を選択した場合,再び衝突する.
  \item
    それぞれの局は0,1,2,3の中から数値をランダムに選択し,そのスロット時間分待機する.
  \item
    3回目の衝突が発生した場合,0から\(2^3 - 1\)の中から数値を選択し,そのスロット時間分待機する.
  \item
    \(i\)回目の衝突が発生した場合,0から\(2^i - 1\)の中から数値を選択し,そのスロット時間分待機する.
    しかし,10回以上の衝突が発生した場合,数値の最大値は1023に固定される.
  \item
    16回連続で衝突が発生した場合,送信を中断し,失敗を通知する.
    その後の回復は上位層に任される.
  \end{enumerate}
\end{itemize}

このアルゴリズムを \textgt{\textbf{べき乗バックオフ(binary exponential backoff)}} と呼ぶ.
送信を行おうとしている局に対して,待ち時間を動的に適応させる事ができる.
連続した衝突が起こるにつれて,ランダムに選ぶ数の範囲を指数関数的に増加させることで,衝突を起こしている局の数が少ないときには短い遅延で処理され,多くの局が衝突しているときには衝突が現実的な時間で解決されることを保証している.
また,バックオフを1023で打ち切ることで,待ち時間が大きくなりすぎることを防いでいる.

衝突が発生しなかった場合,送信者はフレームが相手へ伝送されたものとみなす.
すなわち,CSMA/CDもイーサネットも受信確認を行わない.
これは,誤り率の低い有線チャネルや光ファイバチャネルに適している.
よって,いかなる誤りもCRCによって検出し,上位層で回復しなければならない.

\subsection{イーサネットの性能}\label{ux30a4ux30fcux30b5ux30cdux30c3ux30c8ux306eux6027ux80fd}

クラシック・イーサネットの性能について考える.
べき乗バックオフ・アルゴリズムの完全な解析は非常に複雑であるため,以下の仮定を行う.

\begin{itemize}
\item
  \(k\)個の局が常に送信可能な状態にあり,一定の高い負荷が与えられている.
\item
  各スロットにおいて再送確率は一定である.
\end{itemize}

各局が確率\(p\)で競合しているスロットに転送すると,いずれかの局がそのスロットにチャネルを獲得する確率\(A\)は,いずれか1つの局が転送し(確率\(p\)),他の局が転送しない(確率\(1-p\))という状況であるため,次のようになる.
\begin{align}
  A = kp(1-p)^{k-1}
\end{align}
\(A\)は\(p=1/k\)のとき最大になり,\(k \to \infty\)で\(A \to 1/e\)(\(e\)は自然対数の底)となる.
競合期間がちょうど\(j\)スロットである確率は\(A(1-A)^{j-1}\)であるため,1つの競合あたりの平均スロット数は次の式で与えられる.
\begin{align}
  \sum^{\infty}_{j=0} jA(1-A)^{j-1} = \frac{1}{A}
\end{align}

各スロットは,\(2 \tau\)時間待つので,平均競合期間\(\omega\)は,\(2 \tau / A\)である.
最適な再送確率\(p\)を設定すると,競合スロットの平均数は\(e\)を超えない.
したがって\(\omega\)はせいぜい\(2 \tau e = 5.4 \tau\)である.

平均的なフレームを転送するのに\(P\)秒かかるとすると,多くの局がフレームを送信したい場合,
\begin{align}
  \text{チャネル効率} = \frac{P}{P + 2 \tau / A}
\end{align}
となる.
ここで,2つの局間の最大ケーブル距離がどのように性能に影響するかを考える.
ケーブルが長くなるにつれて,伝搬遅延時間\(\tau\)が長くなり,競合期間も長くなる.
よって,ケーブルを長くするとチャネル効率が低下してしまうため,イーサネットの標準で最大ケーブル長を仕様として定めている.

式をフレームあたりの最適な競合スロット数\(e\)のときについて,フレーム長\(F\),ネットワークの帯域幅\(B\),ケーブルの長さ\(L\),信号の伝達速度\(c\)を用いて書き換えると,\(P=F/B\)であるから,
\begin{align}
  \text{チャネル効率} = \frac{1}{1+2BLe/cF}
\end{align}
となる. 分母の第2項が大きくなると,ネットワークの効率は下がる.
具体的には,ネットワークの帯域幅かケーブルの長さ(\(B\)と\(L\)の積)が増すと,与えられたフレーム長に対するネットワーク効率が下がってしまう.
クラシック・イーサネットは広い帯域幅を持った長距離の通信に最適であるとはいえない.

\subsection{スイッチ式イーサネット}\label{ux30b9ux30a4ux30c3ux30c1ux5f0fux30a4ux30fcux30b5ux30cdux30c3ux30c8}

イーサネットは1本の長いケーブルによって構成されるクラシック・イーサネットとは異なる進化をした.
断線やゆるい接続を探すことの煩わしさを解消するため,それぞれの局を専用のケーブルで中央の
\textgt{\textbf{ハブ(Hub)}} へ接続するという配線パターンが考え出された.
ハブは接続されたすべての通信線を電気的に接続する.
この構成を図4-17(a)(p280)に示す.

\begin{itemize}
\item
  ハブ

  \begin{itemize}

  \item
    配線 : 電話会社のツイスト・ペア・ケーブル

    \begin{itemize}
  
    \item
      多くの建物がすでにこのケーブルで配線されていた.
    \item
      予備のケーブルも豊富であった.
    \end{itemize}
  \item
    ケーブルの最大長はハブから100m(高品質なカテゴリ5を使用した場合200m).
  \item
    局の追加や撤去が簡単となり,ケーブルの断線も容易に発見できるようになった.
  \item
    論理的には1本の長いケーブルで接続する従来のクラシック・イーサネットと等価であるため,通信容量を増加させることはできない.
    局が増えるにつれ1台あたりの容量は減少し,最終的にはLANが飽和してしまう.
  \item
    物理層の機能を提供する.
  \end{itemize}
\end{itemize}

増大する負荷に対応する別の解決法は,スイッチ式イーサネット(switched
Ethernet)である.
このシステムの中心は,図4-17(b)(p280)のように,すべてのポートをつなぐ高速なバックプレーンを備えた
\textgt{\textbf{スイッチ(switch)}} である.

\begin{itemize}
\item
  スイッチ

  \begin{itemize}

  \item
    配線の抜き差しによって簡単に局の追加や撤去ができる.
  \item
    個々のケーブルやポートは,たいてい1つの局にのみ影響を及ぼすため,故障箇所を用意に発見できる.
  \item
    データリンク層を理解する.
  \item
    ハブとは異なり,フレームのあて先になっているポートにしか出力しない.

    \begin{itemize}
  
    \item
      ポートがフレームを受け取ると,スイッチはフレームのイーサネット・アドレスを調べ,あて先ホストがどのポートに接続されているかを判断する.
      そのため,スイッチはポートとアドレスの対応を把握しておく必要がある.
    \item
      その他のポートにはフレームが存在することすらわからない.
    \end{itemize}
  \end{itemize}
\end{itemize}

2つ以上の局,あるいはポートが同時にフレームを送信しようとした場合,どのようなことが起こるか考える.
ハブではすべての局が同一の \textgt{\textbf{衝突ドメイン(collision domain)}}
\footnote{衝突が発生したときにその影響を受ける範囲のこと}
を持ち,転送のスケジューリングにCSMA/CDを用いる.
スイッチでは各ポートは各自の衝突ドメインを持つ.
一般に全二重通信では衝突が起こらず,CSMA/CDも必要ないため,局とポートの両方が他の影響を受けずに通信線へ同時にフレームを送信できる.
ただし,半二重通信の場合は局とポートがCSMA/CDを用いて転送権を争う.

\begin{itemize}
\item
  ハブに対するスイッチのメリット

  \begin{itemize}

  \item
    衝突が起こらないため,回線をより効率的に使用することができる.
  \item
    異なる局から複数のフレームを同時に送信することができる.

    \begin{itemize}
  
    \item
      ハブに比べて桁違いの性能を得ることができる.
    \end{itemize}
  \item
    トラフィックは目的外のポートへ送信されないため,セキュリティが向上する.

    \begin{itemize}
  
    \item
      LANインタフェースには \textgt{\textbf{プロミスキャス・モード(promiscuous mode)}} があり,自分宛てでないフレームも受け取ることができる.
      スイッチを使用した場合はフレームを特定のポートにしか転送しないため,盗聴される可能性が低くなる.
      しかし,本当にセキュリティが必要な場合はトラフィックを暗号化することが望ましい.
    \end{itemize}
  \end{itemize}
\item
  ハブに対するスイッチのデメリット

  \begin{itemize}

  \item
    構造が複雑であり,ハブよりも高価である.

    \begin{itemize}
  
    \item
      しかし,スイッチの価格も下落しているのでハブは絶滅危惧種である.
    \end{itemize}
  \end{itemize}
\end{itemize}

スイッチは各入力ポート上でイーサネット標準のフレームを待っているだけであるので,いくつかのポートをコンセントレーター(集約器)として使うことができる.
図4-18(p281)で上部右隅のポートは1つの局に接続しているのではなく,12ポートのハブに接続している.
フレームがハブに到着すると,それらは衝突やべき乗バックオフを含む通常の方法でイーサネット・フレームとして競合する.
競合に勝ったフレームはハブを経由してスイッチに送られ,他のフレームと同様に処理される.
一度スイッチに入ると,それらは正しい出力線に転送される.
なお,送信先がハブで接続された回線上にいてもよく,その場合フレームはポートにすでに届いているのでスイッチはそれを渡すだけでよい.

\end{document}
